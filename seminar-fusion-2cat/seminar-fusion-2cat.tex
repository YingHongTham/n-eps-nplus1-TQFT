\documentclass[12pt]{article}
\usepackage{amssymb,amsfonts,amsmath,graphicx,mathtools}
\usepackage[alphabetic,y2k,lite]{amsrefs}
\usepackage{fullpage, mystyle}
\usepackage{MnSymbol}

\usepackage{tikz}
\usepackage{tikz-cd}


\newcommand{\Mod}{{\mathcal{M}od}}
\newcommand{\cHom}{{\mathcal{H}\text{om}}}


\begin{document}

\title{Note in preparation for talk for seminar on Fusion 2-Categories, Winter semester 2022, UHH}
\author{Ying Hong Tham}
\date{28 December, 2022}
\maketitle


The main goal of my talk today is to prove that
a finite semisimple 2-category is the
category of finite semisimple modules
over a multifusion category,
and vice versa.


That is, for a semisimple 2-category $\cC$,
there exists a multifusion category $C$
such that

\[
\cC \simeq \Mod_{s.s.}^{fin}(C)
\]

Here $\Mod_{s.s.}^{fin}(C)$,
which we will abbreviate to $\Mod-C$,
stands for finite semisimple right module categories over $C$.

Conversely, for any mutifusion category $C$,
$\Mod-C$ is a semisimple 2-category.



\subsection{Conventions}

Everything is over an algebraically closed field $\kk$
with characteristic 0.

We use different fonts/alphabets for different levels
of structures:

In relation to a 2-category:
\begin{itemize}
\item $\cC,\mathcal{F}$ (caligraphic font): 2-category,
	functor between 2-categories

\item $C,X,Y$ (upper case latin): object of 2-category

\item $f,g$ (lower case latin): 1-morphism;
	we write $\cC(X,Y)$ for the category of morphisms
	from $X$ to $Y$

\item $\eta,\veps,\delta$ (lower case greek): 2-morphism;
	for a 2-morphism $\alpha: f \Rightarrow g: X \to Y$,
	we may write $\alpha \in \cC(X,Y)(f,g)$
	to indicate its sources and targets,
	or simply $\alpha \in \Hom(f,g)$ if the objects are clear
\end{itemize}

In relation to a 1-category:
\begin{itemize}
\item $C,A$ (upper case latin): category, functor between categoreis

\item $a,b,f,g$ (lower case latin): objects in category

\item $\alpha,\beta$ (lower case greek): morphism in category
\end{itemize}

We also compose morphisms from right to left:
in a 2-category $\cC$,
for $\alpha \in \cC(X,Y)(f,f'),
\beta \in \cC(Y,Z)(g,g'),
\gamma \in \cC(X,Y)(f',f'')$,
we write
\[
g \circ f, g \circ f', \ldots : X \to Z
\]
for composition of 1-morphisms,
\[
\beta \circ \alpha: (g \circ f) \Rightarrow (g' \circ f'):
	X \to Z
\]
for horizontal composition of 2-morphisms,
\[
\gamma \cdot \alpha: f \Rightarrow f'' : X \to Y
\]
for vertical composition of 2-morphisms.


We may also omit the composition symbols
if the type of composition is clear
(in particular for composition of 1-morphisms).



In general, if $P$ is a property of a 1-category,
we say that a 2-category $\cC$ is \emph{locally $P$}
if every hom-category $\cC(X,Y)$ satisfies $P$.

\section{Review}

Let us recall some definitions and facts concerning
semisimple 2-categories.
These where covered in more detail in previous talks,
so here we will simply state them without proof.

\subsection{Idempotent completeness, separable monads, splittings}

\begin{definition}
Let $(t, \mu, \eta)$ be a monad on an object $X$
in a 2-category $\cC$.
We say $t$ is \emph{separable} if there is a
$t$-$t$-bimodule section
$\delta: t \Rightarrow t \circ t$ to $\mu$.
\end{definition}

\begin{definition}
Let $r \vdash l: X \to Y$ be an adjunction
with unit $\eta: \id_X \Rightarrow rl$
and counit $\veps: lr \Rightarrow \id_Y$.
We say the adjunction $l \dashv r$ is \emph{separable}
if $\veps$ admits a section.
\end{definition}

Clearly, if an adjunction $l \dashv r$ is separable,
then the monad $rl$ is separable.

\begin{definition}
Let $(t,\mu,\eta)$ be a separable monad on
an object $X \in \cC$.
A \emph{splitting} of $t$ is a separable adjunction
$r \vdash l: X \to Y$
together with an isomorphism
$\psi: rl \simeq t$ as monads on $X$.
\end{definition}

Under the right conditions
(local idempotent completeness of $\cC$), splittings are unique
(see \prpref{p:splitting-unique}).




\begin{proposition}[Uniqueness of splitting]
[\cite{DRfusion}, Theorem A.3.1]
\label{p:splitting-unique}
In a locally idempotent complete 2-category $\cC$,
splittings of a separable monad are unique
up to equivalence.
\end{proposition}

In particular, this holds true when $\cC$
is locally semisimple.

Thought to self:
\cite{DRfusion} proves this by showing that
a splitting is equivalent
to an ``Eilenberg-Moore object'' and also a ``Kleisli object'',
which are in themselves important and interesting objects
characterized by universal properties.
I'd like to have a more direct proof,
somehow constructing an equivalence between any two splittings
directly from the splitting data.
I don't have a full proof, but here's an attempt.
Say for a separable monad $t$ on $X$,
we have two adjunctions $r \vdash l : X \to Y$
and $r' \vdash l' : X \to Y'$
that split $t$.
There is are obvious 1-morphisms $l' \circ r: Y \to Y'$,
$l \circ r': Y' \to Y$,
but it is unlikely that these are equivalences.
A promising candidate for an equivalence
is $l' \circ_t r$, the coequalizer of
$l' \circ t \circ r \Rightarrow l' \circ r$
(here $r$ is a left $t$-module from the counit:
$tr \simeq rlr \Rightarrow r$;
similarly $l'$ is a right $t$-module).
Such a coequalizer does appear in the proof of
\cite{DRfusion}[Theorem A.3.1] anyway,
and I've found that this thought process
makes the consideration of the Eilenberg-Moore and Kleisli
objects less of an ass-pull.


\subsection{Simple objects}


\begin{proposition}[equivalent notions of simple-ness]
Let $\cC$ be a locally finite semisimple and
idempotent complete 2-category,
and let $X \in \cC$ be an object.
Then the following notions of $X$ being simple are equivalent:
\begin{itemize}
\item any subobject $i: Y \to X$ of $X$
 is either 0 ($Y \simeq 0$)
 or ($i$ is) an equivalence;

\item $X$ cannot be written as a non-trivial direct sum,
	i.e. if $X = \boxplus X_i$,
	then $X_i \simeq 0$ for all but one $i$;

\item $\id_X$ is a simple object in $\cC(X,X)$.
\end{itemize}
\end{proposition}


\begin{definition}[(finite) semisimple 2-category]
A 2-category $\cC$ is \emph{semisimple}
if it is:
\begin{itemize}
\item locally semisimple,
\item admits left and right adjoints for every 1-morphism,
\item additive,
\item idempotent complete.
\end{itemize}

It is furthermore \emph{finite semisimple}
if it is also locally finite and
has finitely many equivalence classes of simple objects.
\end{definition}

\begin{thebibliography}{1}

\bibitem{DRfusion} Douglas, Christopher L., and David J. Reutter. ``Fusion
2-categories and a state-sum invariant for 4-manifolds.'' arXiv preprint arXiv:1812.11933 (2018).

\end{thebibliography}



\end{document}
